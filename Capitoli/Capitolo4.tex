\chapter{Telescopi a raggi X}

\section{Introduzione ai telescopi a raggi X}

Un contributo recente e fondamentale all'evoluzione dell'astrofisica si trova negli anni '60 dello scorso secolo, con prime realizzazioni negli anni '70, e riguarda l'approdaggio della banda X allo studio dei corpi celesti. Il motivo per cui si aggiunse così in ritardo questa banda rispetto a quella dell'ottico e dell'infrarosso viste fino ad ora, sta nella difficoltà che energie e frequenze così implicano nel riuscire a fuocheggiare i raggi entranti. Negli anni '60 i raggi X erano rilevabili prevalentemente da rilevatori Geiger e similari, in più senza possibilità di fuocheggiarli in un'area ridotta si avrebbe sia un rapporto segnale-rumore grandissimo, sia una forte limitazione di osservazione a quelle che sono le dimensioni dell'area collettrice del detector. Viste queste notevoli problematiche, in quegli anni Riccardo Giacconi e Bruno Rossi iniziarono a cercare un modo per riflettere e fuocheggiare questi raggi, sfruttando un principio di riflessione già noto all'epoca. I raggi X trovano nei materiali ad elevato numero atomico Z (i metalli) un mezzo otticamente meno denso, i cui indici di rifrazione rispetto ai fotoni in quella banda sono di poco minori a uno. Ricordando la legge di Snell:
\begin{equation}
    \frac{\sin\theta_i}{\sin\theta_r}=\frac{1}{n_{X-ray}}
\end{equation}
Ne consegue che l'angolo di riflessione rispetto al piano di riflessione di questi materiali è leggermente maggiore di quello incidente, per cui è possibile fuocheggiare raggi X tramite sistemi poco curvi posti in direzione radente rispetto quella di incidenza. In questo caso il fenomeno di riflessione e rifrazione è simile a quanto avviene nell'ottico, ci sarà un determinato valore detto \textit{angolo critico} per cui i raggi riflessi sono paralleli al piano del materiale (o meglio al piano che separa i due mezzi a densità ottica diversa). Al di sotto di questo valore non vi è una separazione del fascio in riflessione o rifrazione ma si osserva una riflessione totale. Si ricorda la definizione di questo angolo critico come:
\begin{equation*}
    \cos\theta_c = n_{X-ray} =
    1-\delta \quad \longrightarrow \quad
    \theta_c \simeq \sqrt{2\delta}
\end{equation*}
Dove si è introdotto il parametro $\delta$ per definire lo scarto tra l'indice di rifrazione del mezzo nella banda X e l'unità per pure ragioni di comodità, essendo che tutti gli indici che vengono trattati hanno uno scarto negativo piccolissimo rispetto all'unità.
